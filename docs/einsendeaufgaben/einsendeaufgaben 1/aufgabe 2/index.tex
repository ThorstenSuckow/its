\chapter{Aufgabe 2:  Recherche von aktuellen Angriffsszenarien}

\section{a)}

\textit{Recherchieren Sie im Internet bzgl. aktueller Angriffe auf IT‐Systeme und sicher‐
heitsrelevanter Meldungen, die in den letzten Monaten stattgefunden haben, und geben Sie fünf der gefundenen Angriffe bzw. Meldungen an. }

\vspace{5mm}

\begin{itemize}
    \itemsep0.5em
    \item \textbf{Bybit-Hack} Im Februar 2025 wird die Kryptobörse \textit{Bybit} Opfer eines Hacks, bei denen über 1.5 Milliarden Dollar von nordkoreanischen Hackern erbeutet werden\footnote{
    \url{https://announcements.bybit.com/en/article/incident-update-unauthorized-activity-involving-eth-cold-wallet-blt292c0454d26e9140/}, abgerufen 23.03.2025
    }. Es berichten u.a. Telepolis\footnote{
        \url{https://www.telepolis.de/features/Groesster-Krypto-Diebstahl-aller-Zeiten-Nordkorea-klaut-1-5-Milliarden-Dollar-bei-Bybit-10295169.html}, abgerufen 23.03.2025
    } sowie Spiegel.de \footnote{
        \url{https://www.spiegel.de/netzwelt/web/bybit-nordkoreas-hacker-sollen-milliarden-bei-kryptoboerse-erbeutet-haben-a-529d2692-6be9-4f4b-965f-7b192c7c6f2a}, abgerufen 23.03.2025
    }.
    \item \textbf{Storm-0408} Microsoft berichtet im März 2025\footnote{
    \url{https://www.microsoft.com/en-us/security/blog/2025/03/06/malvertising-campaign-leads-to-info-stealers-hosted-on-github/}, abgerufen 23.05.2025
    } von einem Malware-Angriff über (illegale) Streamingseiten, bei denen eingebettete Werbeanzeigen Schadsoftware auf den Rechner der Benutzer installieren.
    \item \textbf{DDOS-Angriff X.com} Wired.com berichtet im März 2025\footnote{
    \url{https://www.wired.com/story/x-ddos-attack-march-2025/}, abgerufen 23.03.2025
    }über eine Distributed-Denial-of-Service-Attacke gegen X.com (ehemals twitter.com).

    \item \textbf{Fortinet Zero-Day exploit} In einer Warnmeldung vom Januar 2025 berichtet das BSI\footnote{
        \url{https://www.bsi.bund.de/SharedDocs/Cybersicherheitswarnungen/DE/2025/2025-213432-1032.pdf?__blob=publicationFile&v=2}, abgerufen 23.03.2025
    } von einer Zero-Day-Schwachstelle bei FortiOS und FortiProxy, durch die Angreifer Super-Admin-Privilegien erlangen können.
    \item \textbf{Sicherheitsgefahr durch EOL-Versionen von MS Exchange} das BSI berichtet im März 2023\footnote{
    \url{https://www.bsi.bund.de/SharedDocs/Cybersicherheitswarnungen/DE/2024/2024-223466-1032.pdf?__blob=publicationFile&v=7} ,abgerufen 23.03.2025
    } von Sicherheitslücken bei Versionen von MS Exchange Servern, die offiziell keinen Support mehr erfahren (\textit{End-of-Life}), aber trotzdem noch von Institutionen betrieben werden.
\end{itemize}