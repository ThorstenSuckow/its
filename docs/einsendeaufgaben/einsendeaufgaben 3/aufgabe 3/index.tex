\chapter{Aufgabe 3}

\textit{Es seien $x$, $n_1$ und $n_2$ positive ganze Zahlen. Beweisen oder widerlegen Sie
folgende allgemeine Aussage: $(x \ \text{MOD} \ n_1) \ \text{MOD} \ n_2 = (x \ \text{MOD} \ n_2) \ \text{MOD} \ n_1$.}\\

\noindent
\textbf{Beweis:}\\

\noindent
Seien $s_1, s_2, n_1, n_2, r_1, r_2 \in \mathbb{N}$ mit $0 \leq r_1 < n_1$ sowie $0 \leq r_2 < n_2$, so dass

\begin{equation}\notag
    x = s_1 n_1 + r_1 \Rightarrow r_1 = x \ \text{MOD} \ n_1
\end{equation}

\noindent
und

\begin{equation}\notag
x = s_2 n_2 + r_2 \Rightarrow r_2 = x \ \text{MOD} \ n_2
\end{equation}

\noindent
gilt.\\

\noindent
Wir finden leicht Zahlen $x = 5, n_1 = 2, n_2 = 3$, für die wir die Aussage für den allgemeinen Fall widerlegen können, denn:

\begin{equation}\notag
    \begin{split}
        r_1 &= 5 \ \text{MOD} \ 2 = 1\\
        r_2 &= 5 \ \text{MOD} \ 3 = 2
    \end{split}
\end{equation}

\noindent
womit gilt:

\begin{equation}\notag
\begin{split}
(5 \ \text{MOD} \ 2) \text{MOD} \ 3  = 1 \ \text{MOD} \ 3 = 1 \neq 0 = 2 \ \text{MOD} \ 2 = (5 \ \text{MOD} \ 3) \text{MOD} \ 2
\end{split}
\end{equation}


\qed


\section{b)}

\textit{Entwerfen Sie ein Verfahren, mit dessen Hilfe ein (vorgegebener) zufälliger 128‐
Bit‐AES‐Schlüssel k vertraulich und authentisch von A an B gesendet werden
kann. Hierbei können Sie davon ausgehen, dass bei der Übertragung keine Fehler
auftreten. Im Gegensatz zum klassischen Envelope‐System dürfen Sie hier allerdings lediglich einen einzigen 512‐Bit‐Block übertragen.
Hinweis: Achten Sie bei der Kombination der Basismechanismen Verschlüsselung und Signatur auf die Reihenfolge der Anwendung.}\\

\noindent

\begin{enumerate}
    \itemsep0.5em
    \item Bereitstellen des Schlüssel $k$ als 128 Bitfolge
    \item Erstellen der RSA-Signatur von $s= S(k, d_A)$ ($d_A$ = privater Schlüssel von $A$) (\textit{Authentizität})
    \item Verschlüsseln von $k||s$ mit dem öffentlichen Schlüssel von $B$ über \textit{RSA} (Gesamtgröße $\rightarrow$ 512 Bit) (\textit{Vertraulichkeit})
    \item $B$ entschlüsselt $k||s$  mit seinem privaten Schlüssel.
    \item $B$ überprüft die Signatur mit dem öffentlichen Schlüssel von $A$.
    Wenn der Vergleich übereinstimmt, ist die Authentizität garantiert, $k$ kann zum Entschlüsseln verwendet werden.
\end{enumerate}


\section{c)}

\textit{Wie müssen Sie in b) vorgehen, wenn nicht sichergestellt werden kann, dass die
Übertragung fehlerfrei ist?}\\

\noindent
Man könnte mit \textit{Paritätsbits} arbeiten, die eine Überprüfung der empfangenen Bitfolge auf gerade bzw. ungerade Parität ermöglicht.
Schlägt die Paritätsprüfung fehl, kann die Nachricht verworfen werden und ggf. eine erneute Zusendung des Schlüssels angefordert werden.
Das würde bedeuten, dass der 512-Bit-Block faktisch verlängert wird.\\

\noindent
Bei einzelnen Bitfehlern in dem 512-Bit-Block würden Übertragungsfehler beim Entschlüsseln nicht zwingend erkannt werden, wenn es sich um einen RSA-verschlüsselten \textbf{Klartext} handelt - ggfl. würden nur fehlerhafte Werte in der entschlüsselten Nachricht auftauchen, die im besten Fall keinen Sinn ergeben.\\
Allerdings behandeln wir an dieser Stelle eine Signatur und einen Schlüssel, die mit RSA verschlüsselt wurden.
Spätestens bei Anwendung der entschlüsselten Komponenten würden bei der Signaturüberprüfung  (Übertragungs-)Fehler auffallen, da eine Signatur auch \textit{Integrität} sicherstellt - also sind in diesem Fall keine weiteren Maßnahmen nötig.