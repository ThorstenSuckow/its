\chapter{Aufgabe 2}

\textit{Für die Ermittlung des Sicherheitsniveaus können sogenannte ``Was-wäre-wenn-\ldots?``-Fragen sehr hilfreich sein. Formulieren Sie zehn solcher Fragen, um in einem Unternehmen das Informationssicherheitsniveau zu ermitteln.}\\

\noindent
Der BSI-Standard 200-2 empfiehlt in~\cite[110]{BSI200-2}, für die Ermittlung des Schutzbedarfs der Geschäftsprozesse deren Bedeutung für die Institution zu hinterfragen.
Dies kann durch geeignete ``Was-wäre-wenn-?``-Fragen unterstützt werden, wobei es ausreichend sein sollte, diese mit einem ausgewählten Personenkreis - z.B. ausgewiesene Fachexperten der jeweiligen Fachabteilungen, Abteilungs-/Standortleiter - zu sammeln, zu erörtern und basierend auf einer Risikoanalyse, aus der sich ein Verhältnis zwischen Notwendigkeit, Umsetzbarkeit und Kosten/Aufwand ergibt, in die Einstufung des Sicherheitsniveaus einfließen zu lassen.\\

\noindent
Als grobe Leitlinie für die Formulierung der Fragen könnten die \textit{CIA+}-Schutzziele dienen, die auch im Folgenden zur Formulierung der Fragen genutzt werden.

\begin{itemize}
\itemsep0.5em
\item \textbf{Vertraulichkeit} - \textit{Was wäre, wenn \ldots}
\begin{itemize}
\item[] \ldots unternehmensinterne E-Mails von anderen mitgelesen werden?
\item[] \ldots E-Mails an Kunden von anderen mitgelesen werden?
\item[] \ldots Ergebnisse der Projekte / der Stand der Projekte von anderen in Erfahrung gebracht werden?
\item[] \ldots rechtlich geschützte Daten technisch nicht ausreichend gesichert werden?
\end{itemize}
\item \textbf{Integrität} - \textit{Was wäre, wenn \ldots}
\begin{itemize}
\item[] \ldots nicht sichergestellt werden kann, dass die Kunden-/Unternehmenskommunikation geschützt ist?
\item[] \ldots Steuerdaten für die ICS-Systeme gefälscht werden?
\item[] \ldots die Manipulation von Daten erst spät auffällt?
\end{itemize}
\item \textbf{Verfügbarkeit} - \textit{Was wäre, wenn \ldots}
\begin{itemize}
\item[] \ldots unsere interne IT ausfällt?
\item[] \ldots die externe Internetanbindung wegfällt?
\item[] \ldots es zum Löschen geschäftskritischer Daten kommt?
\item[] \ldots nicht schnell genug auf einen Ausfall/Wegfall reagiert werden kann?
\end{itemize}
\item \textbf{Authentizität} - \textit{Was wäre, wenn \ldots}
\begin{itemize}
\item[] \ldots Dritte Zugangsdaten zu unseren internen IT-Bereichen erhalten?
\item[] \ldots private Schlüssel kompromittiert werden?
\item[] \ldots Zertifikate auslaufen oder ungültig werden?
\item[] \ldots sich Dritte als Kunden oder Geschäftspartner ausgeben?
\item[] \ldots die Kompromittierung von Authentifizierungsdaten erst spät auffällt?
\end{itemize}
\end{itemize}

\noindent
Bei der Fragestellung zeigt sich, dass die Zeit eine durchaus kritische Metrik sein kann.

\noindent
Aus den aufgeführten Fragen leiten sich weitere Fragen ab, die direkte Auswirkungen auf den Geschäftsbereich eines Unternehmens haben und deshalb für eine Risikoanalyse und die damit verbundene Verhältnismäßigkeit entsprechender Maßnahmen relevant sein können. Diese sind nachfolgend aufgeführt\footnote{
Richtlinien zur Fragestellung bietet auch der BSI-Standard in seinem Anhang, vgl.~\cite[173 ff.]{BSI200-2}


\begin{itemize}
    \itemsep0.5em
    \item Werden Verbindungsdaten / Infrastruktur / Authentifizierungsdaten offengelegt?
    \item Werden Geschäftsstrategie / Geschäftsgeheimnisse offengelegt?
    \item Wird Verhandlungstaktik / werden Kundenanforderungen offengelegt, sodass Dritte dies zur Abwerbung nutzen könnten?
    \item Können sensible Informationen der Kunden / des Unternehmens unbemerkt zugunsten Dritter geändert werden?
    \item Werden Geschäftsgeheimnisse so aufbewahrt, dass eine Änderung dieser nachhaltig die Wirtschaftlichkeit / Solvenz / Betriebsbereitschaft des Unternehmens gefährdet?
    \item Gilt das Gleiche für Daten der Produktion (ICS)?
    \item Ist unser Tagesgeschäft abhängig von einer funktionierenden IT-Infrastruktur? (Software-/Hardwarefehler, interner/externer Geschäftsbetrieb)
    \item Wie stellt unser Backup-Konzept den Fortbestand der Produktion sicher, wenn Daten wegfallen / die Hardware oder ICS-Systeme ausfallen?
    \item Sind Produktionsdaten interner Systeme ausreichend geschützt, und kann ein Passwortmissbrauch schnell aufgedeckt werden?
    \item Ist unsere Schlüsselinfrastruktur auf eine umgehende Invalidierung kompromittierter Schlüssel ausgelegt?
    \item Können wir bei der elektronischen Kommunikation die Identität der Kommunikationsteilnehmer garantieren?
\end{itemize}