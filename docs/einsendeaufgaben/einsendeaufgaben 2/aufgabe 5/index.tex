\chapter{Aufgabe 5}

\textit{a) Von wem wurden die Common Criteria entwickelt?}\\

Mitte der 1990er Jahre von Kanada, Frankreich, Deutschland, den Niederlanden, dem Vereinigten Königreich (UK) und den USA\footnote{
    \url{https://www.commoncriteriaportal.org/iccc/ICCC_arc/history.htm}, abgerufen am 28.03.2025
}\\

\noindent
\textit{b) Mit welcher Zielsetzung?}\\

Die Common Criteria haben das Ziel, international gültige Kriterien und Regeln für eine gegenseitige Anerkennung von Zertifizierungen in der Informationssicherheit festzulegen und damit der zunehmenden Globalisierung Rechnung zu tragen (vgl.~\cite[85]{ITS2}).\\

\noindent
\textit{c) Was ist ein EVG?}\\

\textit{EVG} steht für \textit{Evaluationsgegenstand} - damit ist im Kontext des IT-Sicherheitsmanagements ein zu prüfendes IT-System oder IT-Produkt zu verstehen (vgl.~\cite[86]{ITS2}).\\

\noindent
\textit{d) Was ist ein PP?}\\

Unabhängig vom Einsatzgebiet eines EVG können \textit{Schutzprofile} (\textit{Protection Profile}, \textit{PP}) definiert werden, die letztendlich \textit{Sicherheitsanforderungen} umfassen, die auf eine EVG-Familie anwendbar sind.
Grundsätzlich geht es also darum, die Sicherheitsanforderungen, die ein EVG erfüllen muss, in einem Katalog von Sicherheitsanforderungen für eine ganze EVG-Familie zusammenzufassen.\\
Diese werden später in spezielle \textit{Sicherheitsvorgaben} für den EVG überführt (vgl.~\cite[219f.]{Eck18}).\\

\noindent
\textit{e) Welche Vertrauenswürdigkeitsstufen kennen die Common Criteria?}\\

Das CC-Zertifikat gilt als Nachweis für die Vertrauenswürdigkeit eines evaluierten Systems (vgl.~\cite[219]{Eck18}).
Die Common Criteria kennen hierbei die folgenden Vertrauenswürdigkeitsstufen, die als \textit{Evaluation Assurance Level} (\textit{EAL}) bezeichnet werden (vgl.~\cite[94f.]{ITS2}):

\begin{itemize}
\itemsep0.5em
\item \textit{EAL1} - functionally tested
\item \textit{EAL2} - structurally tested
\item \textit{EAL3} - methodically tested and checked
\item \textit{EAL4} - methodically designed, tested and reviewed
\item \textit{EAL5} - semiformally designed and tested
\item \textit{EAL6} - semiformally verified design and tested
\item \textit{EAL7} - formally verified design and tested
\end{itemize}

\noindent
Anzumerken ist hierbei, dass das CC-Abkommen vorsieht, dass Zertifikate, die von \textit{EAL1} bis einschließlich \textit{EAL4} ausgestellt wurden, automatisch in allen teilnehmenden Ländern anerkannt werden (vgl.~\cite[96]{ITS2}).
