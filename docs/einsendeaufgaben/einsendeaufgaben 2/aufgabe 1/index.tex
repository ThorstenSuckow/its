\chapter{Aufgabe 1}

\textit{Definieren Sie,
    \begin{itemize}
        \item welcher Wert den Geschäftsprozessen und den geschäftsrelevanten Informationen
        sowie welcher Stellenwert der IT und
        \item welche Bedeutung der Informationssicherheit
        beim Beispielunternehmen zugemessen werden sollte, wie also das Sicherheitsniveau
        einzustufen ist
    \end{itemize}
    Begründen Sie Ihre Auswahl kurz. (Hinweis: Verwenden Sie vorformulierte Einschätzungen)
}\\

\subsection*{Einführung}

Verschiedene Stufen des Sicherheitsniveaus sind in~\cite[36f.]{ITS2} angegeben; diese sind durch die IT-Grundschutz-Methodik\footnote{
\textit{BSI-Standard 200-2},~\cite{BSI200-2}
} wohlbekannt und bieten einen Maßstab für bedarfsgerechte Maßnahmen, um Vertraulichkeit, Verfügbarkeit und Integrität für Geschäftsprozesse, Informationen und IT-Systeme zu gewährleisten (vgl.~\cite[11]{BSI200-2}).\\

Zur Definition von Sicherheitszielen und dem damit verbundenen Sicherheitsniveau empfiehlt die IT-Grundschutz-Methodik zunächst, ``klarzustellen, wie stark die Aufgabenerfüllung innerhalb der Institution von der Vertraulichkeit, Integrität und Verfügbarkeit von Informationen und von der eingesetzten IT und deren sicherem Funktionieren abhängt``, wobei es sinnvoll sei, ``die zu schützenden Grundwerte Verfügbarkeit, Integrität und Vertraulichkeit ausdrücklich zu benennen und eventuell zu priorisieren`` (\cite[24]{BSI200-2}).\\

Dies soll im Folgenden in angemessen kurzer Form gezeigt werden, um im Anschluss daraus ein empfohlenes Sicherheitsniveau für das Beispielunternehmen abzuleiten.

\subsection*{Wert der Geschäftsprozesse, der geschäftsrelevanten Informationen, Stellenwert der IT, Bedeutung Informationssicherheit}

Ohne sehr genaue Kenntnisse bezüglich der Kernfachlichkeit der angegebenen Geschäftsprozesse darf wohl festgestellt werden, dass \textbf{Integrität} eine zentrale Rolle im Geschäftsfeld des Unternehmens spielt: Das Unternehmen sammelt Informationen und bereitet diese in unterschiedlicher Form auf (``nach vorgegebenen Kriterien``, ``Lageberichten``, ``Pressemitteilungen``, ``Presseauswertungen``).
Das Geschäft basiert auf der (digitalen) Verarbeitung und Auswertung von Informationen, wobei die Daten nicht nur bei der Sammlung auf Unverfälschtheit und Echtheit überprüft werden müssen.
Dies fällt in den Aufgabenbereich der jeweiligen Fachabteilungen - das \textbf{Informationssicherheitsmanagement} muss parallel dazu geeignete Maßnahmen treffen, um diese Daten auch für den späteren Gebrauch sicher zu archivieren, um eine Manipulation durch Dritte auszuschließen.
Hierbei spielen neben \textit{präventiven} auch \textit{detektive Sicherheitsmaßnahmen} eine Rolle, um Manipulationen nicht nur zu verhindern, sondern im schlimmsten Fall auch feststellen zu können.
Zu diesen Maßnahmen gehört dann die Absicherung der Server bzw. Datenspeicher, um sie vor unberechtigtem Zugriff und vor Manipulation zu schützen.
In diesem Beispiel wäre dies auch durch bauliche Maßnahmen zu unterstützen, da eine von drei TK-Komponenten von dem Unternehmen direkt betrieben wird.\\

Wie unschwer zu erkennen ist, spielt hier auch das Schutzziel \textbf{Verfügbarkeit} eine wichtige Rolle: Man darf annehmen, dass das Unternehmen seinen Umsatz nicht durch das einmalige Ausliefern dieser - durch hohen personellen Aufwand und entsprechende technische Hilfsmittel - gesammelten Informationen erzielt, sondern die Daten müssen auch für eine spätere Auswertung zur Verfügung stehen, z.B. wenn Kunden Presseauswertungen historisiert anfordern, diese also über einen gewissen Zeitraum gesammelt und dann zusammengetragen werden müssen.
Hierzu müssen geeignete Maßnahmen getroffen werden, um die Daten abzusichern, was durch entsprechende Notfallvorsorge (Backups) unterstützt werden muss.
Vermutlich muss beim Verlust von Daten nicht umgehend reagiert werden (z.B. weil die Daten nicht rund um die Uhr einem Kunden zur Verfügung stehen müssen), aber es muss dennoch sichergestellt werden, dass die Daten bei Verlust innerhalb eines angemessenen Zeitrahmens den Fachabteilungen - und damit letztendlich den Kunden - wieder zur Verfügung gestellt werden können.\\

Die \textbf{Vertraulichkeit} ist als eher \textit{hoch} einzuordnen: Auch wenn das Unternehmen Daten sammelt und aufbereitet, die in der Regel für jeden zugänglich sind (Pressemitteilungen, Internet-Artikel, Reaktionen aus den sozialen Netzwerken), kann eine Analyse und Auswertung Informationen beinhalten, die zwischen Unternehmen und Kunde als vertraulich zu behandeln sind.
Bei der Kommunikation solcher Informationen sollte also auf sichere Übertragungswege zurückgegriffen werden, was auch eine Verschlüsselung der Daten bei direktem Zugriff (geschützter, verschlüsselter Kundenbereich des Internetauftritts, Verschlüsselung der E-Mail-Kommunikation) nach sich zieht.\\

Die o.a. Beurteilung stützt sich überwiegend auf die Informationen, die der Aufgabe hinsichtlich des Kerngeschäftsbereichs des Beispielunternehmens zu entnehmen sind. Darüber hinaus sind Tätigkeiten angegeben, die in modernen Unternehmen heutzutage Usus sind, also elektronisches Ticketsystem, Kundensupport über E-Mail und Telefon, Dienstleistungen im Bereich Internetauftritt und eine standesgemäße IT-Infrastruktur für ein mittelständisches Unternehmen (bei 100 Mitarbeitenden).\\
In Summe lassen sich hier übliche Schutzziele für ein von einer funktionierenden IT-Infrastruktur abhängiges Unternehmen festmachen:
\begin{itemize}
\itemsep0.5em
\item Das Unternehmen bietet digitale Informationsverarbeitung an - Daten und Technik müssen deshalb für das Tagesgeschäft \textbf{hochverfügbar} sein.
\item Die Kundenkommunikation muss in einem Großteil der Fälle \textbf{vertraulich} erfolgen, um Informationen zu laufenden Aufträgen auf Basis der Authentizität der Kommunikationsteilnehmer austauschen zu können.
\item Die \textbf{Integrität} von Daten als ``Nebenprodukt`` des Tagesgeschäfts sollte sichergestellt werden, auch wenn man hier sicherlich weiter abgrenzen kann, da ``einfache Informationsschriften`` als Dienstleistung ggf. weniger schützenswert sind als die Datenanalyse für einen Großkonzern oder eine Behörde.
\end{itemize}\\

\subsection*{Sicherheitsniveau}

Der BSI-Standard 200-2 nennt beispielhafte Sicherheitskriterien zur Bestimmung eines angemessenen Sicherheitsniveaus - diese werden klassifiziert als \textit{normal}, \textit{hoch}, \textit{sehr hoch} (vgl.~\cite[24f.]{BSI200-2})\footnote{
\textit{Münch und Schaumüller-Bichl} geben in~\cite[37]{ITS2} das weitere Schutzziel \textit{niedrig} an. Die Semantik ist hier für den angegebenen Fall bereits ein Ausschlusskriterium (Sicherheitsniveau niedrig: ``Vertraulichkeit von Informationen ist nicht gefordert``).
}.\\

Mit der angegebenen Analyse lässt sich das Sicherheitsniveau für das Unternehmen als \textbf{hoch} einschätzen, was - basierend auf den Kriterien im BSI-Standard - wie folgt begründet wird:

\begin{itemize}
\itemsep0.5em
\item \textbf{Schutz vertraulicher Informationen muss hohen Anforderungen genügen und in sicherheitskritischen Bereichen stärker ausgeprägt sein}:
\item[] Dies wurde in der Analyse mit dem Kerngeschäft (sehr kritisch) und den mit digitalen Dienstleistungen üblich verbundenen Schutzzielen begründet.
\item \textbf{Verarbeitete Informationen müssen korrekt sein, auftretende Fehler müssen erkennbar und vermeidbar sein}:
\item[] Besonders relevant bzgl. der Akkumulierung der Daten, die auch über längere Zeit fortbestehen müssen und deren Integrität stets sichergestellt werden muss, um keine verfälschten Analysen und Auswertungen zu erstellen.
\item \textbf{Es laufen zeitkritische Vorgänge, es werden Massenaufgaben wahrgenommen, die ohne IT-Einsatz nicht zu erledigen sind. Es können nur kurze Ausfallzeiten toleriert werden}:
\item[] Obwohl die Integrität der Daten bei der Analyse stärker gewichtet wurde, gilt auch dieser Punkt für das angegebene Szenario. Das Tagesgeschäft ist von der IT abhängig, es sind nur kurze Ausfallzeiten zu akzeptieren, zumal die Kunden ``großen Wert darauf legen, dass die Dienstleistungen schnell und zuverlässig erfolgen`` (Zitat Aufgabenstellung).
\item \textbf{Der Schutz personenbezogener Daten muss gewährleistet sein, damit der Betroffene nicht in seiner gesellschaftlichen Stellung oder in seinen wirtschaftlichen Verhältnissen beeinträchtigt wird}:
\item[] Dies ergibt sich aus dem Geschäftsfeld des Unternehmens (Auswertungen öffentlicher Meldungen können die soziale Stellung einer Institution widerspiegeln), aber auch daraus, dass kundenbezogene Daten in den Systemen vorgehalten werden. Zwar geht aus dem Szenario nicht klar hervor, ob diese Kundendaten von dem externen Dienstleister verwaltet werden. Es muss aber angenommen werden, dass die Mitarbeitenden des Unternehmens mit diesen Kundendaten täglich in Berührung kommen, und geschäftliche (finanzielle) Informationen deshalb geschützt werden müssen, was das Szenario durch die Anforderung ``vertrauliche Informationen [sollen] aus ihren Unternehmen nicht in die Hände Dritter gelangen`` vorgibt. Allein deshalb dürfen Schutzmaßnahmen hierzu nicht allein auf die externen Dienstleister delegiert werden: Das Kriterium muss im Interesse des Unternehmens von der Leitungsebene vorgegeben sein. Deren Realisierung muss von dem ISB und den IT-Abteilungen überwacht und sichergestellt werden.
\end{itemize}
