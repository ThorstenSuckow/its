\chapter{Aufgabe 3}

\section{a)}

\textit{Welche Phase kennzeichnet den Beginn des Sicherheitsprozesses?}\\

\noindent
Der BSI-Standard definiert, dass zur Erlangung eines angemessenen und ausreichenden Sicherheitsniveaus für die Informationssicherheit in einem Unternehmen ein \textit{geplantes Vorgehen} und eine \textit{adäquate Organisationsstruktur} erforderlich sind; zudem ist eine \textit{Strategie} zur Erreichung der Sicherheitsziele notwendig (vgl.~\cite[20]{BSI200-2}).\\

\noindent
In der Regel fällt die \textit{strategische Ausrichtung} einer Institution in den Verantwortungsbereich der Leitungsebene, weshalb diese den Sicherheitsprozess \textbf{initiieren} muss.
Die operative Durchführung wird dann durch den Informationssicherheitsbeauftragten sichergestellt:

\blockquote[{\cite[20]{BSI200-2}}]{
    ``Die oberste Leitungsebene muss den Sicherheitsprozess initiieren, steuern und kontrollieren. Die Leitungsebene ist diejenige Instanz, die die Entscheidung über den Umgang mit Risiken treffen und die
    entsprechenden Ressourcen zur Verfügung stellen muss.
    [\ldots]
    Die operative Aufgabe ‚Informationssicherheit‘ wird allerdings typischerweise an
    einen Informationssicherheitsbeauftragten (ISB) delegiert.``
}

\noindent
Die Initiierung des Sicherheitsprozesses umfasst neben der Übernahme der Verantwortung durch die Leitungsebene auch die Planung:
\begin{itemize}
    \itemsep0.5em
    \item des Sicherheitsprozesses
    \item[] sowie
    \item zeitlicher und finanzieller Ressourcen
\end{itemize}

\noindent
Außerdem wird eine Entscheidung für eine Vorgehensweise getroffen.\\

\noindent
Im Anschluss wird die \textbf{Leitlinie zur Informationssicherheit} erstellt, wie \textit{Münch und Schaumüller-Bichl} in ihrer Abbildung \textit{Phasen des Sicherheitsprozesses mit IT-Grundschutz} in~\cite[\textbf{Abb. 3.2}, 32]{ITS2} verdeutlichen.
Die Leitlinie hält fest, ``welche Sicherheitsziele und welches Sicherheitsniveau die Institution anstrebt`` (\cite[32]{ITS2}).

\section{b)}

\textit{Wie wird das ``Maximum-Prinzip`` im Rahmen der Schutzbedarfsfeststellung für IT-Systeme definiert?}\\

\noindent
Der Schutzbedarf von Anwendungen in einem IT-System - wobei unter ``Anwendung`` nicht nur Software-, sondern auch Hardware-Komponenten verstanden werden können - wird auf Basis der Schutzziele \textit{Vertraulichkeit}, \textit{Integrität} sowie \textit{Verfügbarkeit} bestimmt.
Für die Bewertung kann eine Skala verwendet werden, die z.B. die Werte \textit{niedrig}, \textit{normal}, \textit{hoch} enthält.\\
Der Schutzbedarf ergibt sich dann aus dem den Sicherheitszielen zugeordneten Maximum: Wird für eine Komponente z.B. der Schutzbedarf bzgl. \textit{Vertraulichkeit} und \textit{Integrität} als \textit{normal} eingestuft, die \textit{Verfügbarkeit} aber als \textit{hoch}, wird die Anwendung entsprechend dem Maximum-Prinzip als insgesamt \textit{hoch schutzbedürftig} eingeordnet (vgl.~\cite[114]{BSI200-2}).

\section{c)}

\textit{Kennzeichnen Sie aus der folgenden Auswahl die sicherheitsrelevanten Bereiche,
die einen hohen oder sehr hohen Schutzbedarf besitzen oder als
sicherheitskritisch eingestuft werden können.}\\

\begin{enumerate}
\itemsep0.5em
\item IT-Systeme mit höherem Schutzbedarf - \textbf{sehr hoch}
\item Kommunikationsverbindungen nach außen - \textbf{hoch}
\item Kommunikationsverbindungen mit hoch schutzbedürftigen Daten - \textbf{sicherheitskritisch}
\item Kommunikationsverbindungen, über die bestimmte Daten nicht transportiert werden dürfen - \textbf{hoch}\footnote{Hier muss nicht in erster Linie auf die Durchsetzung der Sicherheitsziele bei der Nutzung der Kommunikationsverbindung geachtet werden, sondern darauf, dass die Kommunikationsverbindung für manche Fälle erst gar nicht genutzt wird!}
\end{enumerate}

\section{d)}

\textit{Bestimmen Sie die richtige Reihenfolge der Durchführung einer Risikoanalyse!
\begin{enumerate}
\item Schätzung der Eintrittswahrscheinlichkeiten der Bedrohungen
\item Kombination der Eintrittswahrscheinlichkeiten mit dem Schutzbedarf zur Risikoermittlung
\item Herausarbeiten der relevanten Bedrohungen
\item Auswahl adäquater Sicherheitsmaßnahmen für untragbare Risiken
\end{enumerate}
}\\
\vspace{5mm}
\noindent
Die Reihenfolge kann wie folgt festgelegt werden\footnote{
Vgl. hierzu auch \textit{Risikoanalyse nach BSI-Standard 200-3},~\cite[7]{BSI200-3}
}:

\begin{itemize}
\itemsep0.5em
\item[] \textbf{3.} Im ersten Schritt erfolgt die Identifikation und das \textbf{Herausarbeiten relevanter Bedrohungen}.
\item[] \textbf{1.} Im Anschluss daran wird geschätzt, wie hoch die \textbf{Eintrittswahrscheinlichkeit der Bedrohung} ist - ist diese vernachlässigbar, müssen Maßnahmen zur Abwehr an dieser Stelle nicht weiterverfolgt werden.
\item[] \textbf{2.} Danach wird die \textbf{Eintrittswahrscheinlichkeit mit dem Schutzbedarf der Anwendung kombiniert} - ist der Schutzbedarf einer Komponente \textit{sehr hoch}, die Eintrittswahrscheinlichkeit aber \textit{sehr niedrig}, können für diese Bedrohung u.U. weniger Ressourcen verbraucht und an anderer Stelle sinnvoller genutzt werden (z.B. müsste in einem entsprechenden Szenario die Entkoppelung eines Systems aus einem öffentlichen in einen privat zugänglichen Bereich nicht weiterverfolgt werden).
\item[] \textbf{4.} Zum Schluss findet die \textbf{Auswahl adäquater Sicherheitsmaßnahmen für untragbare Risiken} statt - sind die Risiken zu hoch, um die Bedrohung zu vernachlässigen, müssen angemessene Schutzmaßnahmen festgelegt werden.
\end{itemize}

\section{e)}

\textit{Ein IT-System wird zum Betrieb von zwei Anwendungen benötigt. Die erste hat
den Schutzbedarf ``hoch`` bezüglich Vertraulichkeit und ``normal`` bezüglich
Integrität, die zweite hat den Schutzbedarf ``normal`` bezüglich Vertraulichkeit
und ``hoch`` bezüglich Integrität. Welchen Schutzbedarf bestimmen Sie anhand
dieser Informationen für das IT-System?}\\

\noindent
Die Schutzbedarfsfeststellung ist in Tabelle~\ref{tab:schutzbedarf} angegeben.
Da mindestens eine Anwendung in einem Schutzziel einen hohen Schutzbedarf hat, ist das IT-System für das jeweilige Schutzziel insgesamt \textbf{hoch schutzbedürftig}.

\begin{table}[h!]
\setlength{\tabcolsep}{0.5em}
\def\arraystretch{1.5}
\centering
\begin{tabular}{|l|c|c|}
\hline
\textbf{} & \textbf{Vertraulichkeit} & \textbf{Integrität} \\
\hline
\textbf{Anwendung A} & hoch & normal  \\
\hline
\textbf{Anwendung B} & normal & hoch \\
\hline
\hline
\textbf{Schutzbedarf IT-System} & hoch & hoch \\
\hline
\end{tabular}
\caption{Schutzbedarfseinstufung für ein IT-System mit den Anwendungen A und B.}
\label{tab:schutzbedarf}
\end{table}

\section{f)}

\textit{Welche Funktionsträger müssen bei der Entwicklung von Schadensszenarien für
Geschäftsprozesse und Anwendungen unbedingt nach ihrer persönlichen
Einschätzung befragt werden?
\begin{enumerate}
\item Verantwortliche für den Geschäftsprozess oder die Anwendung
\item Geschäftsführer der Institution
\item Benutzer der Anwendung
\item Verwaltungsabteilung der Institution
\end{enumerate}
}\\
\vspace{5mm}

\noindent
Es sollten Verantwortliche und Experten aus dem \textbf{fachlichen} und \textbf{operativen} Umfeld der Geschäftsprozesse und damit verbundenen Anwendungen befragt werden, also insbesondere \textbf{1.}.\\
Mit aufgenommen in die Befragung werden können auch die \textbf{Benutzer der Anwendung} (\textbf{3.}), sofern ``Anwendung`` sowohl institutionsinterne Hardware- als auch Software-Komponenten umfasst und die Anwender wertvolle Hinweise auf den möglichen Schutzbedarf und Bedrohungen bzgl. der Anwendung geben können.
